\documentclass[10pt,letterpaper]{article}
\usepackage[letterpaper,margin=0.5cm]{geometry}
\usepackage[utf8]{inputenc}
\usepackage{amsmath}
\usepackage{amsfonts}
\usepackage{amssymb}
\usepackage{siunitx}
\author{Jeffrey Wubbenhorst}
\title{Chapter 13}

\begin{document}
% 13.1
\maketitle
\begin{enumerate}
\item Any particle in the universe attracts any other particle with a gravitational force whose magnitude is $F=G\frac{m_1m_2}{r^2}$ where $m_1$ and $m_1$ are the masses of the particles $r$ is their separation, and $G=(6.67\times 10^{-11}N\cdot m^2/kg^2)$.  is the gravitational constant. 
\item The gravitational force between extended bodies is found by adding (integrating) the individual forces on individual particles within the bodies. However, if either of the bodies is a uniform spherical shell or a spherically symmetric solid, the net gravitational force it exerts on an external object may be computed as if all the mass of the shell of body were located at its center. 

% 13.2 [graviation and superposition]




\end{enumerate}
\end{document}