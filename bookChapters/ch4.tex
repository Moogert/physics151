\documentclass[10pt,letterpaper]{article}
\usepackage[letterpaper,margin=0.5cm]{geometry}
\usepackage[utf8]{inputenc}
\usepackage{amsmath}
\usepackage{amsfonts}
\usepackage{amssymb}
\usepackage{graphicx}
\usepackage{siunitx}
% I should add some multicols stuff to the final study guide. 
\author{Jeffrey Wubbenhorst}
\title{Chapter 4}
\begin{document}

\maketitle

\begin{itemize}
\item If a particle undergoes a displacement $\Delta \vec{r}$ in time interval $\Delta t$, its average velocity is given as $\vec{v_{avg}}=\frac{\Delta \vec{r}}{\Delta r}$ 


% 4.2
\item The instantaneous velocity $\vec{v}=\frac{d\vec{r}}{dt}=\frac{dx}{dt}\hat{i}+\frac{dy}{dt}\hat{j}+\frac{dz}{dt}\hat{k}$. 

% 4.4 - projectile motion 

\item Projectile motion for an object in flight: 
\begin{itemize}
\item $1x-x_0=(v_0\cos \theta_0)t$ \\ 
\item $ y-y_0=(v_0\sin \theta_0)t-\frac{1}{2}gt^2$ \\ 
\item $v_y=v_0\sin \theta_0=-gt$ \\ 
\item $v_y^2=(v_0\sin \theta_0 ^2)-2g(y-y_0)$ \\ 
\end{itemize}

The trajectory (path) of a particle in projectile motion is parabolic and is given by 
$y=(tan\theta_0)x-\frac{gx^2}{2(v_0\cos \theta_0)^2}$
if $x_0$ and $y_0$ are 0. 

\item The particle's horizontal range $R$ (distance from launch to landing assuming both points are at the same height) is given as $R=\frac{v_0^2}{g}\sin 2\theta_0$

\item The horizontal range $R$ is maximum for a launch angle of \ang{45}

% 4.5

\item A particle is in uniform circular motion if it travels around a circle or a circular arc at constant (uniform) speed. 

\item The magnitude of the centripetal acceleration is given as $a=\frac{v^2}{r}$ 
\item A particle in uniform circular motion will the circumference of the circle in time $T=\frac{2\pi r}{v}$. 

% 4-6

\item When two frames of reference $A$ and $B$ are moving relateive to each other at constant velocity, the velocity of a particle $P$ as measured by an observer in frame $A$ usually differs from that measured from frame $B$. The two measured velocities are related by $\vec{V}_{PA}=\vec{V}_{PB}+\vec{V}_{BA}$ where $\vec{V}_{BA}$ is the velocity of $B$ with respect to $A$. 



\end{itemize}
\end{document}