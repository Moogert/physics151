\documentclass[10pt,letterpaper]{article}
\usepackage[letterpaper,margin=0.5cm]{geometry}
\usepackage[utf8]{inputenc}
\usepackage{amsmath}
\usepackage{amsfonts}
\usepackage{amssymb}
\usepackage{siunitx}
% I should add some multicols stuff to the final study guide. 
\author{Jeffrey Wubbenhorst}
\title{Chapter 9}
\begin{document}
\maketitle


\begin{itemize}

\item The center of mass of a system of $n$ particles is defined to be the point whose coordinates are given by $x_{com}=\frac{1}{M}\sum^n_{i=1}m_i,x_i,y_{com}=\frac{1}{M}\sum^n_{i=1}m_i,y_i,z_{com}=\frac{1}{M}\sum^n_{i=1}m_i,z_i$ or $\vec{r}_{com}=\frac{1}{M}\sum^n_{i=1}m_i\vec{r}_i$ where $M$ is the total mass of the system. 

\item Newton's second law for a system of particles is given by $\vec{F}_{net}=M\vec{a}_{com}$, wher $\vec{F_{net}}$%  is the net force of all the external forces acting on the system 

\item Linear momentum is defined as $\vec{p}=m\vec{v}$. Newton's second law can be written in terms of this momentum: $\vec{F}_{net}=\frac{d\vec{p}}{dt}$. 
\item For a system of particles this becomes $\vec{P}=M\vec{v}_com$ 
\item Applying Newton's second law in momentum form to a particle-like body involved in a collision leads to the impulse-linear momentum theorem: $\vec{p}_f-\vec{p}_i=\Delta\vec{p}=\vec{J}$. 
\item Impulse is defined as $\vec{J}=\int_{t_i}^{t_f}\vec{F}(t)dt$, and has units of $kg\dot m/s$
\item If a system is isolated so that no net external force acts on it, the linear momentum $\vec{P}$, which can be written as $\vec{P}_i=\vec{P}_f$ of the system remains constant 
\item For an inelastic collision of two bodies, the kinetic energy of the two-body system is not conserved (is not a constant). If the system is closed and isolated, the total linear momentum of the system must be conserved, which means that $\vec{p}_{1i}+\vec{p}_{2i}=\vec{p}_{1f}+\vec{p}_{2f}$. 
\item The center of mass of a closed, isolated system of two colliding bodies is not affected by a collision. % there's more to this but whatever I will deal with it later 
\item Elastic collisions in one dimension (for a target body 2 and a incoming body 1) are bound by the equation $v_{1f}=\frac{m_1-m_2}{m1+m_2}v_{1i}$ and $v_{2f}=\frac{2m_1}{m_1+m_2}v_{1i}$
\item If two bodies collide and the motion is not along a single axis (collision isn't head-on), the collision is two-dimensional. If the two-body system is closed and isolated, the law of conservation of momentum applies to the collision and cab be written as $$\vec{P}_{1i}+\vec{P}_{2i}=\vec{P}_{1f}+\vec{P}_{2f}$$ 
If the collision is elastic, the conservation of kinetic energy during the collision gives a third equation $$K_{1i}+K_{2i}=K_{1f}+K_{2f}$$ 

\end{itemize}

\end{document}