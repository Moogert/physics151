\documentclass[10pt,letterpaper]{article}
\usepackage[letterpaper,margin=0.5cm]{geometry}
\usepackage[utf8]{inputenc}
\usepackage{amsmath}
\usepackage{amsfonts}
\usepackage{amssymb}
\usepackage{siunitx}
% I should add some multicols stuff to the final study guide. 
\author{Jeffrey Wubbenhorst}
\title{Chapter 5}
\begin{document}
\maketitle

% 5.1 

% ...I feel like I should have started on this earlier...
\begin{itemize}
\item Newtonian mechanics relate acceleration and forces. 
\item Forces are vector quantities. Their magnitudes are defined in terms of the acceleration they would give the standard kilogram. 
\item A force that accelerates the that standard body by exactly 1 $m/s^2$ is defined to have a magnitude of one N (Newton). 

\item Forces are combined according to the rules of vector algebra. The net force on a body is the vector sum of a the forces acting on the body. 
\item Reference frames in which Newtonian mechanics hold are inertial (or reference) frames. 
\item $\vec{F}_{net}=m\vec{a}$ 
\item $1 \mbox{N}=1 kg\dot m/s^2$

\item Weight it given as $W=mg$ 
 \item A normal force $\vec{F_N}$ is the force on a boy from a surface against which a body presses. Normal force is always perpendicular to surface. 
 \item When a cord is under tension each end of the cord pulls on the body. 
 


\end{itemize}
\end{document}